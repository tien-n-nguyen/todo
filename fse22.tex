\PassOptionsToPackage{table,xcdraw}{xcolor}

\documentclass[sigconf,review,anonymous]{acmart}
\acmConference[ESEC/FSE 2022]{The 30th ACM Joint European Software Engineering Conference and Symposium on the Foundations of Software Engineering}{14 - 18 November, 2022}{Singapore}

%\documentclass[sigconf,review,anonymous]{acmart}
%\acmConference[ESEC/FSE 2021]{The 29th ACM Joint European Software Engineering Conference and Symposium on the Foundations of Software Engineering}{23 - 27 August, 2021}{Athens, Greece}

%\acmConference[ICSE 2022]{The 44th International Conference on Software Engineering}{May 21–29, 2022}{Pittsburgh, PA, USA}

%\documentclass[sigconf,review, anonymous]{acmart}
%\documentclass[sigconf]{acmart}

\usepackage{booktabs}   %% For formal tables:
                        %% http://ctan.org/pkg/booktabs
\usepackage{subcaption} %% For complex figures with subfigures/subcaptions
                        %% http://ctan.org/pkg/subcaption
\usepackage{array}
\usepackage{amsmath,amsfonts}
\usepackage{algorithm}
\usepackage[noend]{algpseudocode}
%\usepackage{algorithmic}
\usepackage{graphicx}
\usepackage{textcomp}
\usepackage{float}
\usepackage{listings}
\usepackage{xspace}
\usepackage{multirow}
\usepackage{amsthm}
\newtheorem{definition}{Definition}
\usepackage{balance}

\usepackage[skins]{tcolorbox}

\usepackage{xcolor,pifont}
\newcommand*\colourcheck[1]{%
	\expandafter\newcommand\csname #1check\endcsname{\textcolor{#1}{\ding{52}}}%
}
\colourcheck{blue}
\colourcheck{green}
\colourcheck{red}

\newtcolorbox{myframe}[2][]{%
  enhanced,colback=white,colframe=black,coltitle=black,
  sharp corners,
  toprule=1.0pt,
  rightrule=0.3pt,
  leftrule=0pt,
  bottomrule=0pt,
  fonttitle=\itshape\scshape\large,
  left=0pt,right=5pt,top=5pt,bottom=3pt,
  attach boxed title to top right={yshift=-0.3\baselineskip-0.4pt,xshift=-5mm},
  boxed title style={tile,size=minimal,left=0.2mm,right=0.5mm,
    colback=white,before upper=\strut},
  title=#2,#1
}

%\newcommand{\code}[1]{{\footnotesize\textsf{#1}}}

\newcommand{\tool}{\textsc{TDConcord}\xspace}

\newtheorem{Definition}{Definition}
\newtheorem{Claim}{Claim}
\newtheorem{Lemma}{Lemma}
\newtheorem{Theorem}{Theorem}

\newcolumntype{L}[1]{>{\raggedright\arraybackslash}p{#1}}
\newtheorem{observation}{Observation}
\newtheorem{property}{Property}
\newcommand{\code}[1]{{\footnotesize\texttt{#1}}}
\usepackage{amsthm}
 \definecolor{dkgreen}{rgb}{0,0.6,0}
\definecolor{gray}{rgb}{0.5,0.5,0.5}
\definecolor{mauve}{rgb}{0.58,0,0.82}
\lstset{frame=tb,
  language=Java,
  aboveskip=3mm,
  belowskip=3mm,
  showstringspaces=false,
  columns=flexible,
  basicstyle={\small\ttfamily},
  numbers=left,
  numberstyle=\tiny\color{gray},
  keywordstyle=\color{blue},
  commentstyle=\color{dkgreen},
  stringstyle=\color{mauve},
  breaklines=true,
  breakatwhitespace=true,
  tabsize=4
}



\begin{document}

%\title[{\tool}: Deep Fault Localization with Code Coverage Representation Learning]{{\tool}: Deep Fault Localization with Code Coverage Representation Learning}

\title[Removal of Obsolete TODO Comments With Dual-Task Learning]{Removal of Obsolete TODO Comments with Dual-Task Learning}


%%%---- AUTHORS BLOCK ------

\setcopyright{none}

\settopmatter{printacmref=false, printfolios=false}

\renewcommand\footnotetextcopyrightpermission[1]{} % removes footnote with conference information in first column


%(1) present information sorted in a way that a CNN can "see" patterns
%discriminating between faulty and non faulty statements more easily;

%(2) identify the actual crash statement to the network;

%(3) present more information to the deep neural network in the form of
%a summary of data dependences for each statement as well as source
%embedding; and

%(4) the suspiciousness of a statement is seen taking into account
%relationships to other statement, as opposed to a statement by itself”



%\input{sections/abstract}
\begin{abstract}
Abstract goes here ...
\end{abstract}


%\settopmatter{printacmref=true, printccs=true, printfolios=false}

%\begin{CCSXML}
%<ccs2012>
%<concept>
%<concept_id>10011007.10011006.10011073</concept_id>
%<concept_desc>Software and its engineering~Software maintenance tools</concept_desc>
%<concept_significance>500</concept_significance>
%</concept>
%</ccs2012>
%\end{CCSXML}

%\ccsdesc[500]{Software and its engineering~Software maintenance tools}

%\keywords{Deep Learning; Automated Program Repair; Context-based Code Transformation Learning}


\maketitle

\section{Introduction}
\label{intro:sec}

Writing TODO comments in source code is a common activity by
developers to denote their pending programming tasks in a project. For
example, a developer adds a TODO comment as in {\em ``TODO: print an
  error message if file not found''}. This comment could serve as a
reminder for himself/herself or a notification for other team members
on a pending task. TODO comments are very useful in improving
program comprehension, helping developers understanding the design
and implementation in source code~\cite{souza-sigdoc05,ying-msr05}.

In an ideal practice, after the developer or others perform the task
by adding/modifying the code to print the error message, (s)he will
remove that TODO comment.  It is reported that developers may have
(partially) completed the task described in the TODO comment, however,
did not update or remove it due to several reasons such as time
constraints or
carelessness~\cite{tdcleaner-fse21,wen-icpc19,icomment-sosp07}. Such
an out-of-date TODO comment is called {\em obsolete} when its
corresponding task was accomplished, however the comment itself is not
removed. Obsolete TODO comments can confuse developers and due to
their inconsistent information. They could make developers lose
confidence and reliability of the
code~\cite{tdcleaner-fse21,icomment-sosp07}. In many cases, the
unreliable documentation becomes misleading and causes software
defects in the future~\cite{icomment-sosp07,lintan-icse11}. In brief,
obsolete TODO comments could reduce software quality including code
comprehension and reliability, and increase software maintenance
costs.

Because manually detecting and removing the obsolete TODO comments is
time consuming and tedious, it is highly desirable to develop an
automated approach to detect/remove them as early as possible before
they mislead developers and cause any consequences. However, there is
very limited work on detecting obsolete TODO comments.


three encoders, i.e., TODO Comment Encoder, Code Change Encoder, and
Commit Message Encoder, to embed TODO comments, code changes, and
commit messages into contextualized vectors respectively. TDCleaner
then learns correlations and interactions between them by optimizing
the final probability score. When it comes to online prediction, for a
given TODO comment, we pair it with the associated code change and
commit message, and fit them into the trained TDClearner model to
estimate their matching score.


\newpage

\balance

%\bibliographystyle{plain}
%\bibliographystyle{ACM-Reference-Format}
\bibliographystyle{ACM-Reference-Format}

\bibliography{References}

\end{document}
