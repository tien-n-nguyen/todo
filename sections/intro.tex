\section{Introduction}
\label{intro:sec}

Writing TODO comments in source code is a common activity by
developers to denote their pending programming tasks in a project. For
example, a developer adds a TODO comment as in {\em ``TODO: print an
  error message if file not found''}. This comment could serve as a
reminder for himself/herself or a notification for other team members
on a pending task. TODO comments are very useful in improving
program comprehension, helping developers understanding the design
and implementation in source code~\cite{souza-sigdoc05,ying-msr05}.

In an ideal practice, after the developer or others perform the task
by adding/modifying the code to print the error message, (s)he will
remove that TODO comment.  It is reported that developers may have
(partially) completed the task described in the TODO comment, however,
did not update or remove it due to several reasons such as time
constraints or
carelessness~\cite{tdcleaner-fse21,wen-icpc19,icomment-sosp07}. Such
an out-of-date TODO comment is called {\em obsolete} when its
corresponding task was accomplished, however the comment itself is not
removed. Obsolete TODO comments can confuse developers and due to
their inconsistent information. They could make developers lose
confidence and reliability of the
code~\cite{tdcleaner-fse21,icomment-sosp07}. In many cases, the
unreliable documentation becomes misleading and causes software
defects in the future~\cite{icomment-sosp07,lintan-icse11}. In brief,
obsolete TODO comments could reduce software quality including code
comprehension and reliability, and increase software maintenance
costs.

Because manually detecting and removing the obsolete TODO comments is
time consuming and tedious, it is highly desirable to develop an
automated approach to detect/remove them as early as possible before
they mislead developers and cause any consequences. However, there is
very limited work on detecting obsolete TODO comments.
