\section{Introduction}
\label{intro:sec}

Writing TODO comments in source code is a common activity by
developers to denote their pending programming tasks in a project. For
example, a developer adds a TODO comment as in {\em ``TODO: print an
  error message if file not found''}. This comment could serve as a
reminder for himself/herself or a notification for other team members
on a pending task. TODO comments are very useful in improving
program comprehension, helping developers understanding the design
and implementation in source code~\cite{souza-sigdoc05,ying-msr05}.

In an ideal practice, after the developer or others perform the task
by adding/modifying the code to print the error message, (s)he will
remove that TODO comment.  It is reported that developers may have
(partially) completed the task described in the TODO comment, however,
did not update or remove it due to several reasons such as time
constraints or
carelessness~\cite{tdcleaner-fse21,wen-icpc19,icomment-sosp07}. Such
an out-of-date TODO comment is called {\em obsolete} when its
corresponding task was accomplished, however the comment itself is not
removed. Obsolete TODO comments can confuse developers and due to
their inconsistent information. They could make developers lose
confidence and reliability of the
code~\cite{tdcleaner-fse21,icomment-sosp07}. In many cases, the
unreliable documentation becomes misleading and causes software
defects in the future~\cite{icomment-sosp07,lintan-icse11}. In brief,
obsolete TODO comments could reduce software quality including code
comprehension and reliability, and increase software maintenance
costs.

Because manually detecting and removing the obsolete TODO comments is
time-consuming and tedious, it is highly desirable to develop an
automated approach to detect/remove them as early as possible before
they mislead developers and cause any consequences. However, there is
still limited work on detecting obsolete TODO comments. Recently, Gao
{\em et al.}~\cite{tdcleaner-fse21} introduced TDCleaner, a deep
learning approach that converts TODO comments, the associated code
changes, and the corresponding commit messages into the contextualized
vectors and uses three encoders to learn correlations between them by
optimizing the final probability score. For prediction, a given TODO
comment {\bf T} is paired with the associated code change {\bf C} and
commit message {\bf M}, and is fit into the trained TDCleaner model
with three encoders to estimate the matching score to determine if the
TODO comment is obsolete or not. The three encoders are trained with
the goal of determining the status {\bf S} of the TODO comment $T$ in
which $S$ is positive when $T$ is resolved (i.e., $T$ is obsolete
because the TODO comment still appears). Basically, TDCleaner is
trained using $\langle T, C, M \rangle$ triples such that the
probability $P(S | \langle T, C, M \rangle)$ is maximized over the
given training dataset. That is, TDCleaner finds $y$ such that $y =
argmax_S P (S | \langle T, C, M \rangle)$. $P(S | \langle T, C, M
\rangle)$ can be seen as the conditional likelihood of predicting the
status $S$ given the input triples $\langle T, C, M
\rangle$~\cite{tdcleaner-fse21}.

While being the first automated tool for cleaning obsolete TODO
comments, TDCleaner still has inherent limitations that cause low
accuracy in detecting obsolete TODO comments.

\underline{First}, while TDCleaner aims to learn the correlations
among $\langle T, C, M \rangle$, those three components: $T$ (TODO
comments), $C$ (code changes), and $M$ (commit messages) are {\em not}
at the same level w.r.t. the task. $T$ is the {\em description} of the
task that needs to be done at the time of the TODO comment being
written. $C$ is the {\em code changes}, which could either (partially)
accomplish or not accomplish the task described in $T$. If $C$
fulfills the task by the time of the commit of the code changes $C$,
$T$ is deemed as {\em osolete} because $T$ is still in the source code
that was checked into the repository. Thus, the {\em concordance
  between $T$ and $C$} is the key to decide if $T$ is obsolete. $T$ is
said to be concordance with $C$ if the code changes $C$ realize the
task described in $T$. In contrast, the commit message $M$ is only a
subodinate for the code changes $C$ in which it is the {\em
  description of the commit} (including the coding task realized in
$C$). In TDCleaner, the concordance among three components in $P(S |
\langle T, C, M \rangle)$ to detect obsolete TODO comments is too
strict. The reason is that if $T$ is in concordance with $C$ ($T
\leftrightarrow C$) and $T$ is not in accordance with $M$ ($T
\nleftrightarrow M$), TDCleaner considers this case as
non-concordance, i.e., $T$ is not obsolete. In fact, this case is
an obsolete TODO comment because the code changes already realized the
task described in $T$, but $T$ still appears in the committed source
code. 
%
Thus, this treatment in TDCleaner {\em reduces recall}.

\underline{Second}, there are cases, in which the commit mesage $M$
describing the {\em entire} commit that includes other changes beside
the code changes $C$ realizing the task described in $T$. Let us use
$M \geq C$ to denote these cases. Therefore, in some instances, we
have $T \leftrightarrow C$ but $M \geq C$ and in some other instances,
we have $T \leftrightarrow C$ and $M \leftrightarrow C$. TDCleaner
would not be able to learn correctly the obsolete cases and would deem
the former case as non-conformance. Thus, TDCleaner will miss those
cases (i.e., {\em lower recall}).
